%\documentclass[11pt,a4paper]{article}

\usepackage[top=12ex, bottom=12ex, left=9ex, right=9ex, showframe=false]{geometry}
\linespread{1.8}
\usepackage{fancyhdr}
\pagestyle{fancy}
\lhead{\fancyplain{}{Yuandong Li}}
\rhead{\fancyplain{}{\leftmark}}
\cfoot{\fancyplain{}{\thepage}}

\usepackage{amsmath, amsthm, amsfonts}
\usepackage{enumerate}
\usepackage{verbatim}
\usepackage{color}

\newtheorem{proposition}{Proposition}[subsection]
\newtheorem{lemma}[proposition]{Lemma}
\newtheorem{theorem}[proposition]{Theorem}
\newtheorem{corollary}[proposition]{Corollary}
\theoremstyle{definition}
\newtheorem{definition}{Definition}[subsection]
\theoremstyle{remark}
\newtheorem{remark}{Remark}
\newtheorem{recall}{Recall}
\newtheorem{example}{Example}
\newtheorem{exercise}{\textcolor{magenta}{Exercise}}[]
\newenvironment{solution}{\begin{proof}[Solution]}{\end{proof}}
\newenvironment{exerlist}{\bfseries{EXERCISE}\setcounter{exercise}{0}\begin{enumerate}[\textcolor{magenta}{Exercise} 1]}{
\end{enumerate}}



\def\defn#1{\begin{definition}#1\end{definition}}
\def\prop#1{\begin{proposition}#1\end{proposition}}
\def\thm#1{\begin{theorem}#1\end{theorem}}
\def\cor#1{\begin{corollary}#1\end{corollary}}
\def\pf#1{\begin{proof}#1\end{proof}}
\def\exc#1{\begin{exercise}#1\end{exercise}}
\def\sol#1{\begin{solution}#1\end{solution}}
\def\rec#1{\begin{recall}#1\end{recall}}
\def\rem#1{\begin{remark}#1\end{remark}}
\def\exer#1#2{\item #1 \sol{#2}}

\def\term#1{\textbf{#1}}
\def\1{\mathbb1}
\def\N{\mathbb{N}}
\def\Z{\mathbb{Z}}
\def\R{\mathbb{R}}
\def\C{\mathbb{C}}
\def\Hom{\operatorname{Hom}}
\def\Aut{\operatorname{Aut}}
\def\Av{\operatorname{Av}}
\def\GL{\operatorname{GL}}




%\begin{document}



\section{Basic Concepts}

Most of the following concepts apply to arbitrary groups.

\subsection{Groups and Subgroups}

\defn{
group, Abelian, conjugate, power, finite, cyclic, subgroup, generate, order, complex product, coset, coset, transversal, p-group. 
}

\prop{
bijective mappings, cyclic groups, subgroup, set-generate, complex product, order of c.p., Lagrange Thm, order of g and G, transversal, complement, Dedekind Identity. 
}

\subsection{Homomorphisms and Normal Subgroups}

\defn{
homomorphism, im, ker, epi/mono/endo/iso/automorphism, normal, simple, factor group, subnormal, section. 
}

\prop{
inverse homo, ker is normal, normal iff, natural homo, Homo Thm, Iso Thm*2, subnormal.
}

\subsection{Automorphisms}

\defn{
AutG, InnG, Z(G), characteristic, X-inv, X-homo. 
}

\prop{
cyclic G/Z(G) implies Abelian G, char is trans. 
}

\subsection{Cyclic Groups}

\defn{$C_n$}

\prop{
subgroups of $\Z$, finite cyclic groups and their subgroups, char, cyclic p-groups, Abelian simple groups. 
}

\subsection{Commutators}

\defn{
commutator subgroup, perfect. 
}

\prop{
comu with homo, smallest s.t. Abelian factor, perfect factor, $[x,yz]=[x,z][x,y]^z$, [X,Y] is normal in $<X,Y>$, Three-subgroups Lem. 
}

\subsection{Products of Groups}

\defn{
internal/external direct product, central product, internal/external semidirect product, involutions, dihedral group. 
}

\prop{
in/ex iso, on Z(G) G' G/N, on normal N, internal structures $G/\bigcap_i N_i \cong G/N_1\times\cdots\times G/N_n$, prod of normal subgs of relative prime order is direct, on elements, central product on G/Z[G], factorization and product of elements in a semiderict product, dihedral groups are semidirect products. 
}

\subsection{Minimal Normal Subgroups}

\defn{
minimal normal subgroups. 
}

\prop{
properties of a min-normal subg, product of min-normal subgs, factorization(structure for Abelian versus uniqueness for non-Abelian). 
}

\subsection{Composition Series}

\defn{
refine normal(rep. subnormal)series to chief(esp. composition) series, solvable group, X-section, X-composition series, X-simple. 
}

\prop{
Jordan-Holder Thm. 
}


%%%%%%%%%% below are solutions of exercises %%%%%%%%

\begin{exerlist}

\exer{hahaha}{jajaja}
	
\end{exerlist}




%\end{document}